\documentclass[../main.tex]{subfiles}

    \begin{document}
    \newpage

%%%%%%%%%%%%%%
    % Volet
    \vspace{15pt}
    %\needspace{20pt} % Réserve de l'espace
\section{Services de mobilité partagée}

    % Block
\begin{block}[Déployer]
    \linespread{0.9}\selectfont % Réduit l'interligne
    \emoji{antenna-bars} % Emoji
    \textit{\small{Développer un réseau de services de mobilité partagée à l'échelle nationale ou régionale, avec un maillage stratégique des pôles d'échange et des quartiers de gare.}}
\end{block}

    \begin{multicols}{2}
    \raggedcolumns
    \small{
Afin de désaturer la demande de stationnement tout en renforçant la visibilité du vélo et de la micro-mobilité dans l’espace public et d'inciter à son usage, \gras{les services de mobilité partagée} doivent être intégrés de manière cohérente aux systèmes de transport en commun. Leur déploiement doit s’inscrire dans une logique d’articulation à \gras{une échelle géographique pertinente}, qu’elle soit \gras{régionale}~–~en particulier en connexion avec les réseaux TER~–~ou \gras{nationale}.
    \\\\
Ainsi, la gestion d'un service de vélo en libre-service pourrait être assurée par \gras{les Autorités Organisatrices de la Mobilité (AOM) régionales}~–~à l'image du syndicat mixte en charge de l'intermodalité Hauts-de-France Mobilités~–, en concertation avec les établissements publics de coopération intercommunale (EPCI), les communes et les opérateurs de transport. Cette coordination permettrait d’assurer un maillage stratégique des quartiers de gare cyclables.
    \\\\
En parallèle, l’intégration de flottes \gras{de vélos ou de trottinettes électriques en flotte libre} (ou semi-libre) au sein de plateformes de \gras{\textsl{Mobility as a Service} (MaaS)} viendrait simplifier l’accès des utilisateurs et assurer une expérience fluide. Cette approche permet notamment de faciliter la réservation, d’offrir une disponibilité en temps réel et de garantir une interconnexion optimisée avec les autres modes de transport. \gras{Une coordination stratégique avec les opérateurs privés} pourrait ainsi permettre de prioriser certains points d’intérêt territoriaux, en particulier les quartiers de gare, afin d’assurer une garantie de service.
    }
    \end{multicols}

    \bigskip
    \noindent
    \begin{tabular}{@{}m{0.3\textwidth} m{0.65\textwidth}@{}}
    \gras{\fontsize{40pt}{40pt}\selectfont
75~\%
} & 
    \small{
C'est la part de voyageurs combinant l'usage du train avec les services de vélo ou de trottinette en libre-service, avec ou sans station, et qui, en l'absence de cette offre de mobilité cyclable, auraient opté pour la voiture sur l'ensemble de leur déplacement.
    }
    \end{tabular}

    \end{document}