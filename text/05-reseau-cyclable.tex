\documentclass[../main.tex]{subfiles}

    \begin{document}
    \newpage

%%%%%%%%%%%%%%%%%%%%%%%%%%%%%%%%%%%%%%%%%%
    % Thématique mobilité
    \vspace{15pt}
    \noindent
    \Large
\textbf{\textcolor{UGEblue}{~Thématique~2~:}}
    \\
\colorbox{UGEblue}{\textbf{\textcolor{white}{~Requalifier les espaces publics~}}}
    \vspace{15pt}

%%%%%%%%%%%%%%
    % Volet
    \vspace{15pt}
    %\needspace{20pt} % Réserve de l'espace
\section{Réseau cyclable structurant}

    % Block
\begin{block}[Développer]
    \linespread{0.9}\selectfont % Réduit l'interligne
    \emoji{bicycle} % Emoji
    \textit{\small{Renforcer la connectivité des quartiers de gare grâce à un réseau intercommunal cyclable structurant pour étendre et améliorer l'accès aux pôles d'échange.}}
\end{block}

    \begin{multicols}{2}
    \raggedcolumns
    \small{
Le développement \gras{d’un réseau cyclable structurant} améliore l’accessibilité aux pôles d’échange et renforce le potentiel d'attractivité des gares. Ces infrastructures doivent être aménagées de manière \gras{dense et continue}, tout en garantissant des conditions \gras{de sécurité}, \gras{de lisibilité} et \gras{de confort} optimales. À cet égard, la priorité doit être accordée aux axes structurants des quartiers de gare, en privilégiant des aménagements cyclables distincts du trafic motorisé.
    \\\\
Au-delà de l'amélioration de l'accès aux gares, nos analyses mettent en évidence l’importance de développer \gras{des liaisons cyclables intercommunales et régionales} de manière à relier les gares et d’étendre ces connexions aux territoires situés au-delà des quartiers de gare élargis. Il apparaît que les cyclistes intermodaux sont enclins, au cours de leurs déplacements, à parcourir \gras{de plus longues distances} lorsque le tracé proposé est plus agréable. Un tel maillage cyclable permettrait facilement de se reporter \gras{vers des gares mieux desservies}, pour éviter les correspondances et pour améliorer \gras{les opportunités de connexion}. De surcroît, \gras{une meilleure cyclabilité} contribue à rendre les déplacements cyclables socialement et démographiquement \gras{plus inclusifs}.
    \\\\
Dans ce contexte, l'impulsion donnée \gras{aux initiatives de «~RER Vélo~»}, comme le réseau lillois \textsl{Vélo Plus}, pose les bases d'\gras{une meilleure hospitalité} des territoires. La question de \gras{la généralisation} de ces projets pensés comme un réseau de transport en commun se pose~: il conviendrait d'étendre ces infrastructures dans les centralités urbaines de la région et de veiller à \gras{une meilleure intégration} avec le système ferroviaire.
    }
    \end{multicols}

    \end{document}