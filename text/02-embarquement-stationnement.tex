\documentclass[../main.tex]{subfiles}

    \begin{document}
    \newpage

%%%%%%%%%%%%%%
    % Volet
    \vspace{15pt}
    %\needspace{20pt} % Réserve de l'espace
\section{Stratégies de dépose et d'emport}

    % Block
\begin{block}[Équiper]
    \linespread{0.9}\selectfont % Réduit l'interligne
    \emoji{p-button} % Emoji
    \textit{\small{Mettre en place une politique de stationnement sécurisé et accessible dans les quartiers de gare étendus ainsi qu'une politique d'embarquement des véhicules compacts.}}
\end{block}


    \begin{multicols}{2}
    \raggedcolumns
    \small{
Déployer \gras{des infrastructures de stationnement sécurisées} constitue une condition incontournable pour favoriser l'intermodalité. Il s'agit ainsi de proposer une offre de stationnement suffisamment \gras{capacitaire à proximité immédiate} des pôles d'échange et \gras{densément implantée au sein des quartiers de gare élargis}. Par ailleurs, il est essentiel d'adapter ces infrastructures à la diversité modale en intégrant \gras{les types de vélo} et \gras{d'engin de déplacement personnel (EDP)}, à propulsion humaine et électriques.
    \\\\
La question de \gras{l'embarquement des véhicules compacts} représente également un enjeu central. Notre enquête indique que les cyclistes intermodaux, en particulier les usagers de la trottinette électrique et du vélo pliant, privilégient une mobilité «~porte-à-porte~» à la fois \gras{vers puis depuis les gares de départ et de destination}. L'installation de consignes sécurisées ne se présente pas comme une solution adaptée à leurs pratiques~: \gras{une politique d'embarquement} facilité et spécifiquement adapté à ces véhicules légers répondrait aux besoins exprimés pour les «~derniers kilomètres~».
    \\\\
Dans cette perspective, \gras{la Loi d’Orientation des Mobilités (LOM)} a amorcé des avancées en imposant l'obligation d'équiper les gares les plus fréquentées en stationnement vélo. Cependant, ces efforts supplémentaires sont nécessaires pour se conformer \gras{aux exigences réglementaires}, mais aussi pour développer et mailler des espaces de stationnement adaptés à ces véhicules (ré)émergents et \gras{au-delà du parvis de la gare}.
    }
    \end{multicols}

    \bigskip
    \noindent
    \begin{tabular}{@{}m{0.1\textwidth} m{0.85\textwidth}@{}}
    \gras{\fontsize{40pt}{40pt}\selectfont
5
} & 
    \small{
C'est la disponibilité médiane des espaces de stationnement vélo situés dans les espaces publics des quartiers de gare, accessibles dans un rayon de trois kilomètres. Parmi ceux-ci, 134 gares sur les 318 en service dans la région ne disposent d'aucun emplacement dédié.
    }
    \end{tabular}

    \end{document}