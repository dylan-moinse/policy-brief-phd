\documentclass[../main.tex]{subfiles}

    \begin{document}
    \newpage

%%%%%%%%%%%%%%%%%%%%%%%%%%%%%%%%%%%%%%%%%%
    % Thématique gouvernance
    \vspace{15pt}
    \noindent
    \Large
\textbf{\textcolor{UGEblue}{~Thématique~4~:}}
    \\
\colorbox{UGEblue}{\textbf{\textcolor{white}{~Établir un cadre organisationnel coordonné~}}}
    \vspace{15pt}

%%%%%%%%%%%%%%
    % Volet
    \vspace{15pt}
    %\needspace{20pt} % Réserve de l'espace
\section{Lieux stratégiques}

    % Block
\begin{block}[Intégrer]
    \linespread{0.9}\selectfont % Réduit l'interligne
    \emoji{magnifying-glass-tilted-left} % Emoji
    \textit{\small{Inscrire et réévaluer les périmètres fonctionnels des quartiers de gare dans les référentiels stratégiques d’aménagement et de mobilité.}}
\end{block}

    \begin{multicols}{2}
    \raggedcolumns
    \small{
Alors que les quartiers de gare mis en relation jouent un rôle central dans l'organisation des flux et des formes urbaines, leur périmètre d'influence est encore \gras{trop souvent sous-estimé} dans les orientations stratégiques de développement territorial. La plupart des plans stratégiques définissent ces espaces sur la base d’\gras{une accessibilité piétonne restreinte}, alors que leur rayonnement s’étend bien au-delà en présence d'\gras{un environnement marchable et cyclable} de qualité.
    \\\\
Notre étude a révélé que la délimitation \gras{réellement accessible (isochrones)} des quartiers de gare est \gras{25 fois plus vaste} qu'habituellement envisagé. Pour aligner la planification urbaine sur cette réalité, il convient de réévaluer et d'élargir les périmètres fonctionnels des quartiers de gare, en les considérant comme des projets de territoire. Ces ajustements doivent s’appuyer sur \gras{des données spatiales et cartographiques} précises, afin de mieux appréhender l’influence effective des gares sur leur environnement et d'orienter les décisions d’aménagement.
    }
    %\\\\
\begin{center}
    \includegraphics[width=0.7\columnwidth]{figures/policy-brief-isochrone-amiens.png}
    \label{isochrone-amiens}
    \vspace{0cm}
    \begin{flushright}
            \scriptsize{\textcolor{darkgray}{Auteur~: Dylan Moinse (2025)}}
    \end{flushright}
\end{center}

    \end{multicols}

    \end{document}