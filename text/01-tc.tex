\documentclass[../main.tex]{subfiles}

    \begin{document}
    \newpage

%%%%%%%%%%%%%%%%%%%%%%%%%%%%%%%%%%%%%%%%%%
    % Thématique mobilité
    \vspace{15pt}
    \noindent
    \Large
\textbf{\textcolor{UGEblue}{~Thématique~1~:}}
    \\
\colorbox{UGEblue}{\textbf{\textcolor{white}{~Faciliter les connexions intermodales~}}}
    \vspace{15pt}

%%%%%%%%%%%%%%
    % Volet
    \vspace{15pt}
    %\needspace{20pt} % Réserve de l'espace
\section{Attractivité du réseau ferroviaire}

    % Block
\begin{block}[Améliorer]
    \linespread{0.9}\selectfont % Réduit l'interligne
    \emoji{bullet-train} % Emoji
    \textit{\small{Garantir une haute qualité du niveau de service en transport en commun par une offre fréquente.}}
\end{block}

    \begin{multicols}{2}
    \raggedcolumns
    \small{
L’attractivité du réseau ferroviaire repose sur une offre de transport en commun structurée autour \gras{d’une cadence soutenue} et \gras{d’une amplitude horaire étendue}, éléments déterminants pour maximiser l’usage du train. Nos analyses démontrent que la fréquence et la plage de service des réseaux de TGV et de TER, \gras{en semaine} comme \gras{le week-end}, constituent des leviers stratégiques pour favoriser un report modal vers la marche, le vélo ou la micro-mobilité en combinaison avec les systèmes de transport en commun.
    \\\\
Les pôles d’échange bénéficiant d’une fréquence accrue influencent significativement \gras{les itinéraires} empruntés par ces usagers, disposés \gras{à allonger leur trajet} et à contourner des gares plus proches afin de profiter d’un niveau de service et d'une connectivité supérieurs. À cet égard, les gares qui s'inscrivent dans \gras{des territoires périurbains}, bien qu’offrant \gras{le plus fort potentiel} pour attirer de nouveaux voyageurs, se heurtent à une desserte encore insuffisante.
    \\\\
Dans cette optique, le déploiement des \gras{Services express régionaux métropolitains (SERM)} s’impose comme un levier stratégique pour renforcer la fréquence des dessertes ferroviaires. Mais un projet de cette taille exige parallèlement \gras{une intensification généralisée} de la densité des circulations ferroviaires aux niveaux régional et national.
    }
    \end{multicols}

    \bigskip
    \noindent
    \begin{tabular}{@{}m{0.3\textwidth} m{0.65\textwidth}@{}}
    \gras{\fontsize{40pt}{40pt}\selectfont
20~\%
} & 
    \small{
C'est le degré d'influence du nombre de dessertes en TGV et en TER sur la fréquentation des gares de la région~: il s’agit du critère le plus déterminant.
    }
    \end{tabular}

    \end{document}